\documentclass{article}

\usepackage{amsmath}
\usepackage{fullpage}
\usepackage{graphicx}
\usepackage{hyperref}
\usepackage{float}
\usepackage{listings}
\usepackage[parfill]{parskip}
\usepackage{titlesec}
\setcounter{secnumdepth}{-1}

\titlespacing\section{0pt}{12pt plus 4pt minus 2pt}{0pt plus 2pt minus 2pt}
\titlespacing\subsection{0pt}{12pt plus 4pt minus 2pt}{0pt plus 2pt minus 2pt}
\titlespacing\subsubsection{0pt}{12pt plus 4pt minus 2pt}{0pt plus 2pt minus 2pt}

\title{BYLAWS OF BLUEPRINT}

\author{\\ \sc BLUEPRINT EXECUTIVE COMMITTEE}
\date{Ratified: April 6, 2017}

\begin{document}

\maketitle

\section{I. ELECTIONS}

\subsection{1. Nominations}
\textbf{1.} Executive Committee must collect nominations for all executive roles starting at least two weeks before the day of an election. 

\textbf{2.} Eligible members may be nominated for and run for multiple positions but they are locked into the first one they win. 

\textbf{3.} A nominee must run for a position unless they withdraw their nomination before a date and time set by the Executive Committee. 

\textbf{4.} Nominations outside the nomination period may be added at the discretion of the Executive Committee with the consent of the nominee.

\subsection{2. Election Rules}

\textbf{1.} Order of elections for each role is the constitutional order of contested positions, then constitutional order of non-contested positions.

\textbf{2.} Members who are graduating or intend to become permanently inactive (i.e. they do not intend to be active in Blueprint for the rest of their college career) cannot vote during their final semester. However, if they intend to return to Blueprint after a period of inactivity, then they are eligible to vote.

\textbf{3.} In the case that there are more than 2 candidates running for a single position, the Executive Committee will implement instant runoff voting. A candidate is selected if they receive the majority of votes in their favor (\textgreater 50 \%). If not, the candidate with the least votes will be eliminated and the votes that went to the eliminated candidate will be allocated to the next highest candidate chosen by the voter. This will continue until a candidate receives a majority of the votes.

\textbf{4.} In the case that 2 candidates or less run for a position, a candidate will be elected if they receive a majority of the votes. If no candidate receives a majority vote, the candidate(s) shall leave the room to allow for group discussion that lasts at most Number of Contestants $\times 5$ minutes. After this group discussion, the group shall recast votes. This will repeat until a candidate receives the majority.  

\subsection{3. Election Procedures}

\textbf{1.} All candidates for the role will leave the room at the start of the election for that role. 

\textbf{2.} Candidates will enter the room one at a time and give a 3 minute presentation, or 5 minutes if the candidate is running for presidency. After each presentation, the candidate will leave the room, allowing for a group discussion that lasts at most 5 minutes.

\textbf{3.} After each presentation, spend 3 minutes, or 5 minutes if candidate is running for presidency, for an individual Question and Answer (Q&A) session.

\textbf{4.} For contested positions, there will be a group Q&A session after each candidate has completed their individual presentation and Q&A session. The total length of the group Q&A session is Number of Contestants $\times 3$ minutes.

\textbf{5.} All candidates will leave the room, allowing for a group discussion that lasts at most Number of Contestants $\times $ minutes. 

\textbf{6.} The Executive Committee will then distribute ballots where all eligible members will be able to vote once for: a specific candidate, No Confidence, or Abstain. Any Abstain votes do not get counted towards the total for that position. 

\end{document}